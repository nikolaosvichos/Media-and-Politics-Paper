% Options for packages loaded elsewhere
% Options for packages loaded elsewhere
\PassOptionsToPackage{unicode}{hyperref}
\PassOptionsToPackage{hyphens}{url}
\PassOptionsToPackage{dvipsnames,svgnames,x11names}{xcolor}
%
\documentclass[
  12pt,
  letterpaper,
  DIV=11,
  numbers=noendperiod]{scrartcl}
\usepackage{xcolor}
\usepackage[top=2.5cm,left=2.5cm,right=2.5cm,heightrounded]{geometry}
\usepackage{amsmath,amssymb}
\setcounter{secnumdepth}{4}
\usepackage{iftex}
\ifPDFTeX
  \usepackage[T1]{fontenc}
  \usepackage[utf8]{inputenc}
  \usepackage{textcomp} % provide euro and other symbols
\else % if luatex or xetex
  \usepackage{unicode-math} % this also loads fontspec
  \defaultfontfeatures{Scale=MatchLowercase}
  \defaultfontfeatures[\rmfamily]{Ligatures=TeX,Scale=1}
\fi
\usepackage{lmodern}
\ifPDFTeX\else
  % xetex/luatex font selection
  \setmainfont[]{Latin Modern Roman}
  \setsansfont[]{Latin Modern Roman}
\fi
% Use upquote if available, for straight quotes in verbatim environments
\IfFileExists{upquote.sty}{\usepackage{upquote}}{}
\IfFileExists{microtype.sty}{% use microtype if available
  \usepackage[]{microtype}
  \UseMicrotypeSet[protrusion]{basicmath} % disable protrusion for tt fonts
}{}
\makeatletter
\@ifundefined{KOMAClassName}{% if non-KOMA class
  \IfFileExists{parskip.sty}{%
    \usepackage{parskip}
  }{% else
    \setlength{\parindent}{0pt}
    \setlength{\parskip}{6pt plus 2pt minus 1pt}}
}{% if KOMA class
  \KOMAoptions{parskip=half}}
\makeatother
% Make \paragraph and \subparagraph free-standing
\makeatletter
\ifx\paragraph\undefined\else
  \let\oldparagraph\paragraph
  \renewcommand{\paragraph}{
    \@ifstar
      \xxxParagraphStar
      \xxxParagraphNoStar
  }
  \newcommand{\xxxParagraphStar}[1]{\oldparagraph*{#1}\mbox{}}
  \newcommand{\xxxParagraphNoStar}[1]{\oldparagraph{#1}\mbox{}}
\fi
\ifx\subparagraph\undefined\else
  \let\oldsubparagraph\subparagraph
  \renewcommand{\subparagraph}{
    \@ifstar
      \xxxSubParagraphStar
      \xxxSubParagraphNoStar
  }
  \newcommand{\xxxSubParagraphStar}[1]{\oldsubparagraph*{#1}\mbox{}}
  \newcommand{\xxxSubParagraphNoStar}[1]{\oldsubparagraph{#1}\mbox{}}
\fi
\makeatother


\usepackage{longtable,booktabs,array}
\usepackage{calc} % for calculating minipage widths
% Correct order of tables after \paragraph or \subparagraph
\usepackage{etoolbox}
\makeatletter
\patchcmd\longtable{\par}{\if@noskipsec\mbox{}\fi\par}{}{}
\makeatother
% Allow footnotes in longtable head/foot
\IfFileExists{footnotehyper.sty}{\usepackage{footnotehyper}}{\usepackage{footnote}}
\makesavenoteenv{longtable}
\usepackage{graphicx}
\makeatletter
\newsavebox\pandoc@box
\newcommand*\pandocbounded[1]{% scales image to fit in text height/width
  \sbox\pandoc@box{#1}%
  \Gscale@div\@tempa{\textheight}{\dimexpr\ht\pandoc@box+\dp\pandoc@box\relax}%
  \Gscale@div\@tempb{\linewidth}{\wd\pandoc@box}%
  \ifdim\@tempb\p@<\@tempa\p@\let\@tempa\@tempb\fi% select the smaller of both
  \ifdim\@tempa\p@<\p@\scalebox{\@tempa}{\usebox\pandoc@box}%
  \else\usebox{\pandoc@box}%
  \fi%
}
% Set default figure placement to htbp
\def\fps@figure{htbp}
\makeatother


% definitions for citeproc citations
\NewDocumentCommand\citeproctext{}{}
\NewDocumentCommand\citeproc{mm}{%
  \begingroup\def\citeproctext{#2}\cite{#1}\endgroup}
\makeatletter
 % allow citations to break across lines
 \let\@cite@ofmt\@firstofone
 % avoid brackets around text for \cite:
 \def\@biblabel#1{}
 \def\@cite#1#2{{#1\if@tempswa , #2\fi}}
\makeatother
\newlength{\cslhangindent}
\setlength{\cslhangindent}{1.5em}
\newlength{\csllabelwidth}
\setlength{\csllabelwidth}{3em}
\newenvironment{CSLReferences}[2] % #1 hanging-indent, #2 entry-spacing
 {\begin{list}{}{%
  \setlength{\itemindent}{0pt}
  \setlength{\leftmargin}{0pt}
  \setlength{\parsep}{0pt}
  % turn on hanging indent if param 1 is 1
  \ifodd #1
   \setlength{\leftmargin}{\cslhangindent}
   \setlength{\itemindent}{-1\cslhangindent}
  \fi
  % set entry spacing
  \setlength{\itemsep}{#2\baselineskip}}}
 {\end{list}}
\usepackage{calc}
\newcommand{\CSLBlock}[1]{\hfill\break\parbox[t]{\linewidth}{\strut\ignorespaces#1\strut}}
\newcommand{\CSLLeftMargin}[1]{\parbox[t]{\csllabelwidth}{\strut#1\strut}}
\newcommand{\CSLRightInline}[1]{\parbox[t]{\linewidth - \csllabelwidth}{\strut#1\strut}}
\newcommand{\CSLIndent}[1]{\hspace{\cslhangindent}#1}



\setlength{\emergencystretch}{3em} % prevent overfull lines

\providecommand{\tightlist}{%
  \setlength{\itemsep}{0pt}\setlength{\parskip}{0pt}}



 


\usepackage{setspace}
\linespread{1.2}
\usepackage{changepage}
\usepackage{longtable}
\usepackage{nameref}
\usepackage{hyperref}
\usepackage{booktabs}
\usepackage{tabularray}
\usepackage{float}
\KOMAoption{captions}{tableheading}
\makeatletter
\@ifpackageloaded{caption}{}{\usepackage{caption}}
\AtBeginDocument{%
\ifdefined\contentsname
  \renewcommand*\contentsname{Table of contents}
\else
  \newcommand\contentsname{Table of contents}
\fi
\ifdefined\listfigurename
  \renewcommand*\listfigurename{List of Figures}
\else
  \newcommand\listfigurename{List of Figures}
\fi
\ifdefined\listtablename
  \renewcommand*\listtablename{List of Tables}
\else
  \newcommand\listtablename{List of Tables}
\fi
\ifdefined\figurename
  \renewcommand*\figurename{Figure}
\else
  \newcommand\figurename{Figure}
\fi
\ifdefined\tablename
  \renewcommand*\tablename{Table}
\else
  \newcommand\tablename{Table}
\fi
}
\@ifpackageloaded{float}{}{\usepackage{float}}
\floatstyle{ruled}
\@ifundefined{c@chapter}{\newfloat{codelisting}{h}{lop}}{\newfloat{codelisting}{h}{lop}[chapter]}
\floatname{codelisting}{Listing}
\newcommand*\listoflistings{\listof{codelisting}{List of Listings}}
\makeatother
\makeatletter
\makeatother
\makeatletter
\@ifpackageloaded{caption}{}{\usepackage{caption}}
\@ifpackageloaded{subcaption}{}{\usepackage{subcaption}}
\makeatother
\usepackage{bookmark}
\IfFileExists{xurl.sty}{\usepackage{xurl}}{} % add URL line breaks if available
\urlstyle{same}
\hypersetup{
  pdftitle={Coming of Age Under Trump},
  colorlinks=true,
  linkcolor={black},
  filecolor={Maroon},
  citecolor={black},
  urlcolor={black},
  pdfcreator={LaTeX via pandoc}}


\title{Coming of Age Under Trump}
\usepackage{etoolbox}
\makeatletter
\providecommand{\subtitle}[1]{% add subtitle to \maketitle
  \apptocmd{\@title}{\par {\large #1 \par}}{}{}
}
\makeatother
\subtitle{The Effect of First Electoral Exposure in a Trump Election on
Long-Term Support for Radical Right Attitudes}
\author{Nikolaos Vichos\textsuperscript{}}
\date{December 19, 2025}
\begin{document}
\maketitle
\begin{abstract}
\begin{spacing}{0.95}
How do early political experiences during moments of democratic strain shape long-term support for radical-right attitudes? This paper examines whether coming of age politically in the context of the 2016 Trump election—characterized by nativist, authoritarian, and populist cues—durably alters individuals’ ideological trajectories. Using a sharp regression discontinuity design (RDD), I compare Americans whose first eligible presidential election was 2012 to those whose first eligible election was 2016, leveraging the age-based cutoff in the 2020 ANES. The regression discontinuity (RD) results reveal little evidence of the predicted effects and where, effects do appear, they run largely counter to expectations. Across most subgroups, there is no detectable shift in support for nativist, authoritarian, or populist attitudes. The few significant estimates point in the opposite direction: individuals first eligible to vote in 2016 tend to express lower, not higher, support for radical-right attitudes, particularly among Republicans and the full sample. Independents show a modest increase in authoritarianism, but this pattern is small, inconsistent, and sensitive to model specification. Taken together, the results suggest a partisan asymmetry that contradicts standard accounts of radical-right mobilization, raising the possibility that early exposure to Trump’s norm-breaking campaign may have triggered a form of resistance or retrospective distancing rather than ideological adoption. These findings provide new insight into how political generations are formed under disruptive conditions while raising broader questions about the fragility and conditionality of democratic commitments in the electorate.
\end{spacing}
\end{abstract}


\pagenumbering{gobble}

\newpage{}

\thispagestyle{empty}

\tableofcontents

\newpage{}

\pagenumbering{arabic}

\section{\texorpdfstring{\textbf{Introduction}}{Introduction}}\label{introduction}

The 2016 U.S. presidential election marked a rupture in American
political life. Supporters chanted ``Lock her up!'' as protestors were
forcibly removed from rallies, while not only political opponents but
also journalists and the media were routinely framed as enemies of the
people. For many young voters, this was their first political memory as
adults. If early political experiences are formative, what are the
downstream consequences of coming of age during an election defined by
persistent attacks against the liberal democratic values protecting
media freedom? This leads to the central question of this paper:

\begin{adjustwidth}{1em}{1em}
\emph{Does entering political life during a disruptive and polarizing election shape long-term support for liberal democratic attitudes toward media freedom?}
\end{adjustwidth}

To answer this question, I use a sharp regression discontinuity design
(RDD) leveraging the age-based cutoff for voting eligibility in the 2012
and 2016 presidential elections. By comparing individuals just eligible
to vote in 2012 with those just eligible in 2016, I isolate the causal
effect of entering the electorate during a conventional electoral
context versus one characterized by intensified rhetoric against freedom
of the press.

\begin{quote}
Across most subgroups, the regression discontinuity estimates show no
evidence that first-time exposure to the 2016 election increased support
for radical-right attitudes. Where significant effects do appear, they
point in the opposite direction: Republicans and the full sample exhibit
\emph{lower} levels of nativism and authoritarianism, while independents
show only a small, non-robust increase in authoritarianism. Overall, the
results run counter to prevailing expectations about
impressionable-years sensitivity to radical-right cues.
\end{quote}

\subsection{Importance}\label{importance}

\begin{quote}
Understanding whether and how liberal democratic values protecting media
freedom (hereinafter referred to as `media attitudes'), defined here as
the combination of nativist, authoritarian, and populist attitudes,
might take root is central to assessing the long-term resilience of
liberal democracy. Young voters are particularly receptive to their
political environment when they first engage with democratic
institution; early adulthood thus exerts disproportionate influence on
long-term political attitudes and behaviors
(\citeproc{ref-jennings2014}{Jennings and Niemi 2014};
\citeproc{ref-ghitza2023}{Ghitza, Gelman, and Auerbach 2023}).
Therefore, studying how this cohort responds to periods of elite norm
violation can offer critical insights into the conditions under which
radical-right attitudes can emerge and an erosion of democratic
attitudes might gain traction.
\end{quote}

This study can also contribute to understanding the `stickiness' of
elite cues during early adulthood. Such top-down cues play a central
role in shaping how citizens interpret political conflict, particularly
when elites signal that violations of democratic norms are permissible
or necessary (\citeproc{ref-druckman2024}{Druckman 2024};
\citeproc{ref-clayton2021}{Clayton et al. 2021}). Although such cues
influence the public broadly, their impact might be amplified among
young adults. Their heightened sensitivity to their political context
makes early adulthood a critical window for understanding how exposure
to elite norm breaking may anchor long-term orientations toward
radical-right attitudes.

Finally, the importance of this study extends beyond academia: if
exposure to norm violations during early adulthood can strengthen
support for radical-right attitudes and weaken democratic commitments,
the long-term costs of elite transgression may be generational as well
as institutional. Democratic erosion might thus unfold subtly, through
shifts in what young citizens come to view as acceptable in a democracy.
If radical-right attitudes take root precisely when political identities
are forming, the effects may endure long after the precipitating crisis
has passed. This possibility raises the question of whether democracy
can survive if its newest citizens come of age during moments of
radical-right.

\section{Literature Review}\label{literature-review}

This paper draws on two strands of literature to examine how coming of
age under a norm-breaking election may shape support for radical-right
attitudes. The first explores how early adulthood functions as a
sensitive period for political and civic development, with political
attitudes and behaviors formed during this time often persisting
throughout adulthood. Within that context, first-time electoral
experiences play a notable role, forming durable patterns of engagement.
Secondly, recent research on democratic backsliding highlights the
importance of elite cues in shaping public attitudes and commitments to
democratic norms, particularly within the Trump context. Taken together,
these strands provide a framework for analyzing how exposure to the 2016
Trump election may have affected the political development of young
voters.

\subsection{Early Adulthood and Future Political
Behavior}\label{early-adulthood-and-future-political-behavior}

Existing research has underscored the enduring impact of childhood and
early adulthood on later-life outcomes: one of the main domains in which
this is most evident is political behavior.

Mannheim (\citeproc{ref-mannheim1928}{1928}) introduces the concept of
`fresh contact,' arguing that initial encounters with political
authority and conflict during this time help crystallize lasting
political worldviews. Recent empirical work supports this idea.
Neundorf, Smets, and García-Albacete (\citeproc{ref-neundorf2013}{2013})
use German panel data to show that political interest develops most
dynamically during adolescence and early adulthood, stabilizing around
age 25. Similarly, Dinas (\citeproc{ref-dinas2013}{2013}), studying
reactions to the Watergate scandal, finds that young adults adjust their
political evaluations more flexibly than older cohorts by placing less
weight on prior attitudes, suggesting an openness to new information.

A related line of research explores how generations differ in their
long-term political behavior based on when they came of age. Ghitza,
Gelman, and Auerbach (\citeproc{ref-ghitza2023}{2023}) find that
political preferences are largely shaped by events encountered between
ages 14 and 24, and that cohort-defining moments like the Kennedy,
Reagan, or Trump years create stable partisan divides. Longitudinal data
confirm that political orientations established in this period often
endure: Jennings and Niemi (\citeproc{ref-jennings2014}{2014}), drawing
on decades of panel data, find that attitudes formed in early adulthood
tend to persist across life, especially for individuals who engaged
politically early on.

\subsubsection{First Electoral
Experiences}\label{first-electoral-experiences}

One of the clearest applications of the formative nature of early
adulthood lies in the act of voting for the first time. Dinas
(\citeproc{ref-dinas2012}{2012}) finds that voting in a first election
increases the likelihood of voting in subsequent elections, even decades
later. This finding is echoed by Franklin
(\citeproc{ref-franklin2004}{2004}) , who argues that the overall
`salience' of a young person's first election (whether it begins with a
high- or low-turnout context) leaves a footprint in their participation
trajectory. These findings reinforce the idea that initial political
experiences, like exposure to and participation in salient elections,
become reference points that influence later engagement.

Taken together, this literature positions early adulthood as a formative
and fragile window, where social contexts and political experience leave
lasting marks on civic behavior. Nevertheless, less is known about how
this sensitivity interacts with moments of democratic crisis,
particularly those marked by surges in nativist, authoritarian, and
populist rhetoric. This paper contributes to this literature by
examining whether coming of age during a period marked by institutional
conflict and democratic norm erosion shapes long-term support for
radical right attitudes. In doing so, it connects theories of
developmental susceptibility with contemporary concerns about democratic
backsliding and elite norm violation.

\subsection{Trump 2016 and the Trickling-Down of Radical-Right
Attitudes}\label{trump-2016-and-the-trickling-down-of-radical-right-attitudes}

The 2016 U.S. presidential election marked an important rupture, both in
tone and behavior, that distinguished it sharply from previous
elections. In addition to being ``one of the most polarizing elections
in U.S. history'', pitting two extremely ideologically opposed
candidates, the 2016 election also stands out particularly due to
candidacy of Donald Trump (\citeproc{ref-bekafigo2019}{Bekafigo et al.
2019}).

\subsubsection{A radical-right
candidacy?}\label{a-radical-right-candidacy}

Trump's campaign style was in many ways unprecedented. His political
strategy heavily depended on activating deep-rooted anxieties: rather
than tempering the Party's message to appeal to broader electorates,
Trump positioned himself against both party elites and marginalized
groups, embracing anti-immigration rhetoric and rejecting political
correctness.

\begin{quote}
Most importantly for the purposes of this paper, his campaign was
characterized by a stark presence of nativist, authoritarian, and
populist ideology. Existing work by Cremer
(\citeproc{ref-cremer2023}{2023}) highlights this, detailing how Trump's
campaign combined these three elements, as exemplified through slogans
such as ``build the wall'' or ``drain the swamp''. Interestingly, these
ideological features were also present on the demand-side: Knuckey and
Hassan (\citeproc{ref-knuckey2022}{2022}) find that authoritarian
predispositions were powerful predictors of white vote choice in 2016.
Authoritarianism exerted a stronger influence in the 2016 presidential
than in any prior election, including the 2016 House election,
suggesting that Trump's campaign style served as a triggering mechanism.
\end{quote}

\subsubsection{Elite cues and anti-democratic
attitudes}\label{elite-cues-and-anti-democratic-attitudes}

While the 2016 election marked a clear breach in elite behavior, the
broader effects of this rupture also depend on how citizens respond.
Findings from the literature on democratic erosion suggest that elite
norm violations can destabilize democratic institutions. Citizens rarely
act as the first movers in democratic erosion; they respond to elite
behaviors, internalizing their cues about what is politically
permissible or necessary.

Druckman (\citeproc{ref-druckman2024}{2024}) offers a comprehensive
framework for understanding how democratic backsliding unfolds through
the interaction of elites, societal actors, and citizens. Elites are
uniquely positioned to violate laws and norms, often under the guise of
defending democracy itself, while also making decisions with an eye
toward how the public will respond. Thus, in Druckman's view, citizens
function as latent legitimators, responding to elite behavior and
helping either to normalize or resist democratic erosion. Similarly,
Clayton et al. (\citeproc{ref-clayton2021}{2021}) use a panel survey
experiment to examine the effect of exposure to Trump's statements
questioning the results of the 2020 presidential election. Their finding
that exposure to these statements eroded trust in the election's
legitimacy and decreased confidence in elections among his supporters
broadly suggest that anti-democratic elite cues can `trickle-down,'
impacting public attitudes toward democratic norms.

This study builds on these insights by asking how the 2016 shift in
elite rhetoric affected young voters who came of age during this period.
While existing work documents the democratic consequences of Trump's
presidency, less is known about how exposure to this political moment
shaped long-term political development. This paper addresses that gap by
examining whether coming of age under Trump altered long-term
radical-right attitudes.

\subsection{Theory and Hypotheses}\label{theory-and-hypotheses}

Building on these insights, this section introduces this paper's
theoretical framework and hypotheses. It outlines how early political
socialization, partisan sensitivity, and the ideologically asymmetric
effects of political context shape long-term liberal democratic
attitudes.

\subsubsection{H1: First-Time
Sensitivity}\label{h1-first-time-sensitivity}

Clearly, early adulthood is politically formative: experiences during
this period can leave long-lasting marks on political behavior. This
paper applies this logic to the context of the 2016 presidential
election; unlike earlier cohorts, first-time voters in 2016 `came of
age' politically during a time characterized by norm-breaking elite
rhetoric, defined by the presence of nativist, authoritarian, and
populist features. Given the norm-eroding context, I argue that the
process of attitude formation becomes distorted at the moment it is most
sticky.

This leads to the central claim tested in this study, reflecting both
the sensitivity of early political experiences and the unique nature of
the 2016 election:\footnote{``Positive effect'' refers to an increase in
  support}

\begin{adjustwidth}{1em}{1em}
\emph{H1: Being a first-time voter in the 2016 election will have a positive effect on long-term support for radical-right attitudes ('first-time sensitivity' hypothesis).}
\end{adjustwidth}

\subsubsection{H2: Non-Partisan
Insensitivity}\label{h2-non-partisan-insensitivity}

A growing body of research suggests that political partisanship shapes
how individuals interpret and respond to their political environment.
Whereas partisan identifiers, especially strong ones, tend to be more
psychologically and behaviorally sensitive to political stimuli,
self-identified independents are generally less responsive.

Mandel and Omorogbe (\citeproc{ref-mandel2014}{2014}) find that
partisans report higher life satisfaction under favorable political
conditions and lower satisfaction under unfavorable ones, while
independents express intermediate levels. This `favorability effect'
suggests that partisans are more emotionally attuned to political
developments whereas independents are less susceptible to affective
spillover from the political context. In terms of political behavior,
Magleby, Nelson, and Westlye (\citeproc{ref-magleby2011}{2011}) find
that independents vote at significantly lower rates than partisans. This
turnout gap---nearly 20 percentage points in the 2008
election---suggests lower overall political involvement.

These findings suggest that independents are less likely to internalize
elite cues associated with polarized political contexts and thus may
have been less susceptible to the cues of the 2016 Trump campaign. This
leads to my second hypothesis:

\begin{adjustwidth}{1em}{1em}
\emph{H2: The effect will be stronger among partisans (Republicans or Democrats) than among independents ('non-partisan insensitivity' hypothesis).}
\end{adjustwidth}

\subsubsection{H3: Asymmetric
Sensitivity}\label{h3-asymmetric-sensitivity}

My final hypothesis builds on the ``asymmetric sensitivity'' hypothesis
proposed Mandel and Omorogbe (\citeproc{ref-mandel2014}{2014}), positing
that Republicans are more affectively responsive to their political
environment than Democrats: their retrospective and prospective
well-being was more strongly conditioned by whether the political
environment was favorable or not. Democrats, by contrast, were less
reactive.

This pattern is consistent with research showing that conservatives are
more likely to attend to, and be affected by, affectively charged or
threatening information. For example, Joel, Burton, and Plaks
(\citeproc{ref-joel2014}{2014}) find that conservatives not only
anticipate stronger emotional reactions to negative outcomes but also
report experiencing them more intensely.

These findings, in addition to the fact that the radical-right elite
cues came primarily from the Republican side imply that Republican
first-time voters may have been more deeply affected by the political
context of 2016 and, rather than resisting Trump's radical-right cues,
they may have internalized them as acceptable
(\citeproc{ref-cremer2023}{Cremer 2023};
\citeproc{ref-knuckey2022}{Knuckey and Hassan 2022}) . This leads to my
final hypothesis:

\begin{adjustwidth}{1em}{1em}
\emph{H3: Among partisans, the effect will be stronger among Republicans ('asymmetric sensitivity' hypothesis).}
\end{adjustwidth}

\section{Research Design}\label{research-design}

This study uses a sharp RDD to identify the long-term causal impact of
being first eligible to vote during a `conventional' versus
`unconventional' period and political environment on voters'
radical-right attitudes. The core idea is that young people who were
just old enough to vote in the 2012 U.S. presidential election (i.e.,
age 18 by November 2012) experienced a fundamentally different initial
political environment than those who were just too young. While the
former `came of age' politically during the 2012 election between Barack
Obama and Mitt Romney, the latter were first eligible to vote in 2016,
during the Trump era.

This design translates into a cutoff at age 26 in 2020 (the year for
which data is used): respondents who were 26 or older in 2020 were just
old enough to vote in 2012, while those under 26 were not. The
assumption underlying the RDD is that, within a narrow bandwidth around
this cutoff, the `treatment' (being born after November 6, 2012) is
as-good-as-randomly assigned. The local comparison around the threshold
thus simulates an experiment where this `treatment' is assigned by birth
timing. For more details on the research design, including the specific
regression formula and index construction, please refer to the
\href{https://github.com/nikolaosvichos/Online-Appendix-Coming-of-Age-Under-Trump}{online
appendix.}

\subsection{Data and Index
Construction}\label{data-and-index-construction}

The analysis draws on the 2020 American National Election Studies
(\citeproc{ref-anes}{ANES 2021}) cross-sectional survey, which offers a
large, nationally representative sample and includes a relatively broad
set of items on radical-right attitudes.

To capture individual-level support for radical-right attitudes, I
constructed three indices, using a total of 28 items: nativism (6),
authoritarianism (9), populism (13). For each core feature, I first
assessed the internal consistency of these items\footnote{Even though
  the authoritarian and populist items each captured a coherent
  underlying construct (Cronbach's α \textgreater{} 0.7), the nativist
  items were just below the commonly accepted threshold, at 0.67,
  raising doubts about the internal consistency of the index and
  potentially explaining some of the irregularities in the results
  (\citeproc{ref-taber2018}{Taber 2018})}, used factor analysis to
extract a single latent factor and rescaled the factor scores to a 0--1
range, where higher values indicate stronger support for radical-right
values.

\subsection{Analysis}\label{analysis}

My analysis is organized around running separate regressions for
different subgroups to test the three hypotheses:

\begin{longtable}[]{@{}
  >{\centering\arraybackslash}p{(\linewidth - 2\tabcolsep) * \real{0.2520}}
  >{\centering\arraybackslash}p{(\linewidth - 2\tabcolsep) * \real{0.7480}}@{}}
\caption{Specific sub-dataset used for each hypothesis}\tabularnewline
\toprule\noalign{}
\begin{minipage}[b]{\linewidth}\centering
Hypothesis
\end{minipage} & \begin{minipage}[b]{\linewidth}\centering
Subgroup
\end{minipage} \\
\midrule\noalign{}
\endfirsthead
\toprule\noalign{}
\begin{minipage}[b]{\linewidth}\centering
Hypothesis
\end{minipage} & \begin{minipage}[b]{\linewidth}\centering
Subgroup
\end{minipage} \\
\midrule\noalign{}
\endhead
\bottomrule\noalign{}
\endlastfoot
H1

(first-time sensitivity) & Full dataset \\
H2

(non-partisan insensitivity) & Two sub-datasets: partisans
(self-identified Democrats and Republicans), and independents \\
H3

(asymmetric sensitivity) & Two sub-datasets: Democrats and
Republicans \\
\end{longtable}

I thus expect the causal effect to be stronger for (i) partisans than
independents and (ii) Republicans than Democrats. For each subgroup and
each index, I estimated two model specifications:

\begin{longtable}[]{@{}cc@{}}
\caption{Model specifications}\tabularnewline
\toprule\noalign{}
Hypothesis & Subgroup \\
\midrule\noalign{}
\endfirsthead
\toprule\noalign{}
Hypothesis & Subgroup \\
\midrule\noalign{}
\endhead
\bottomrule\noalign{}
\endlastfoot
Simple Model & No controls \\
Control Model & Gender, education, and income as control variables. \\
\end{longtable}

\section{Results}\label{results}

This section presents the findings of this paper. I begin by providing
some example-visualizations of the discontinuity near the 26-year-old
threshold for the full sample and for the Republican
subgroup.\footnote{The running variable (age) axis (horizontal) in these
  visualizations was narrowed down to an 18-60 years range (rather than
  the original 18-60 years) for a clearer depiction. Because of the high
  number of discontinuity visualizations (15) and their general lack of
  visual clarity, I only present the full sample and the Republican
  subgroup as examples. For all 15 visualizations, please refer to the
  \href{https://github.com/nikolaosvichos/Online-Appendix-Coming-of-Age-Under-Trump}{online
  appendix.}} I then present the results, analyzing them by hypothesis.

\subsection{Discontinuities}\label{discontinuities}

Figure~\ref{fig-discontinuityplot-full} visualizes the discontinuity in
at the threshold for the full-sample. Upon visual inspection, is no
clear discontinuity for none of the three outcomes, and the overall
clutter of the scatterplot makes it difficult to discern any patterns.
Though there does appear to be a slight discontinuity in nativism, the
confidence intervals (CIs) are largely overlapping.

\begin{figure}[H]

\centering{

\pandocbounded{\includegraphics[keepaspectratio]{Vichos_MediaPaper_files/figure-pdf/fig-discontinuityplot-full-1.pdf}}

}

\caption{\label{fig-discontinuityplot-full}Discontinuity in Media
Attitudes at Age 26 (Full Sample, First Degree Polynomial)}

\end{figure}%

Figure~\ref{fig-discontinuityplot-republicans} shows the discontinuity
at the threshold for the Republican subgroup. In this case, there does
appear to be a discontinuity in authoritarianism and populism attitudes
when crossing the 26-year threshold (though CIs slightly overlap).
Interestingly, it appears that crossing the threshold has a positive
effect on support for these attitudes. This suggests an opposite effect
to that predicted originally: `coming of age' politically under Trump
leads to lower support for authoritarianism and populism among
Republicans.

\begin{figure}[H]

\centering{

\pandocbounded{\includegraphics[keepaspectratio]{Vichos_MediaPaper_files/figure-pdf/fig-discontinuityplot-republicans-1.pdf}}

}

\caption{\label{fig-discontinuityplot-republicans}Discontinuity in Media
Attitudes at Age 26 (Republican Subgroup, First Degree Polynomial)}

\end{figure}%

\subsection{RDD Results}\label{rdd-results}

The results are visualized in Figure~\ref{fig-coefplot-results}. The
points represent the estimated causal effect at the cutoff, and the
horizontal lines show the associated 95\% CIs.

\begin{figure}[H]

\centering{

\pandocbounded{\includegraphics[keepaspectratio]{Vichos_MediaPaper_files/figure-pdf/fig-coefplot-results-1.pdf}}

}

\caption{\label{fig-coefplot-results}Coefficient plot for Nativism
showing estimated effects and confidence intervals.}

\end{figure}%

The results, summarized below, are quite perplexing.\footnote{The
  direction of the results should be interpreted in accordance with the
  wording of the first-time sensitivity hypothesis: a positive effect
  means an increase in nativist, authoritarian, or populist attitudes
  among those who voted for the first time in 2016.

  However, in reality, these results have been inverted. This is because
  of the rdrobust R package (\citeproc{ref-calonico2025}{Calonico,
  Cattaneo, and Farrell 2025}). In the package, the threshold is
  ``crossed'' from left to right---hence, normally, a positive effect
  would suggest an increase in nativist, authoritarian, or populist
  attitudes among those who voted for the first time in \emph{2012.}
  However, for clearer comprehension, the results have been reversed,
  and the would indices indicate \emph{decreasing} support for each of
  the three ideological features. For more details on index
  construction, please refer to the online appendix.

  Due to the perplexing nature of these results, I ran a number of
  `rule-of-thumb' tests to confirm that the direction of the index is
  the intended one, such as checking if it is negatively or positively
  correlated with items used in its construction or other variables that
  we would expect to be correlated with the outcomes in a specific way
  (for example, higher education levels correlated with lower support
  for authoritarianism). These tests confirmed that the direction of the
  indices is the intended one.} Across most-subgroups, results are
statistically insignificant. The only exceptions are nativism for the
full sample (only for the model without controls) and authoritarianism
for the Republican and independent subgroups (only for the model without
controls for among independents). Support for authoritarianism increased
among independents; however there was no statistically signficant
increase among partisans. The two other findings are even more
surprising, however, suggesting the exact opposite from what predicted:
both Republicans and the entire sample of respondents who voted for the
first time in 2016 are \emph{less} authoritarian or nativist.

\begin{longtable}[]{@{}
  >{\centering\arraybackslash}p{(\linewidth - 2\tabcolsep) * \real{0.1088}}
  >{\centering\arraybackslash}p{(\linewidth - 2\tabcolsep) * \real{0.8912}}@{}}
\caption{Summary of RD results}\tabularnewline
\toprule\noalign{}
\begin{minipage}[b]{\linewidth}\centering
Hypothesis
\end{minipage} & \begin{minipage}[b]{\linewidth}\centering
Reject the Null?
\end{minipage} \\
\midrule\noalign{}
\endfirsthead
\toprule\noalign{}
\begin{minipage}[b]{\linewidth}\centering
Hypothesis
\end{minipage} & \begin{minipage}[b]{\linewidth}\centering
Reject the Null?
\end{minipage} \\
\midrule\noalign{}
\endhead
\bottomrule\noalign{}
\endlastfoot
\begin{minipage}[t]{\linewidth}\centering
H1

(first-time sensitivity)\\
\strut
\end{minipage} & \begin{minipage}[t]{\linewidth}\centering
\emph{No.} The only significant result is a negative effect in support
for nativist attitudes, which contradicts H1's prediction. Additionally,
the result is not robust to the inclusion of controls.\\
\strut
\end{minipage} \\
\begin{minipage}[t]{\linewidth}\centering
H2

(non-partisan insensitivity)\\
\strut
\end{minipage} & \begin{minipage}[t]{\linewidth}\centering
\emph{No.} The only significant result is that independents who voted
for the first time in 2016 were \emph{more} authoritarian, which
contradicts H2's prediction of a stronger effect among partisans.
Additionally, the result is not robust to the inclusion of controls\\
\strut
\end{minipage} \\
H3

(asymmetric sensitivity) & \emph{No.} The only significant result is
that Republicans who voted for the first time in 2016 were \emph{less}
authoritarian, which contradicts H3's prediction, despite suggesting
some sort of asymmetric insensitivity. \\
\end{longtable}

\section{Conclusion and Discussion}\label{conclusion-and-discussion}

This paper set out to examine whether the timing of democratic
initiation---specifically, becoming a first-time voter during the 2016
U.S. presidential election---shapes long-term support for radical-right
attitudes. Drawing on theories of political socialization, elite
cue-taking, and asymmetric partisan sensitivity, I tested whether
`coming of age' politically in a moment of nativist, authoritarian, and
populist elite rhetoric would leave a measurable imprint on voter
attitudes.

The findings are perplexingly counter-intuitive. First, independents
voting for the first time in 2016 indicated \emph{stronger} support for
authoritarianism than partisans, contradicting existing work suggesting
that independents are less sensitive to their political context. Even
more surprisingly, `coming of age' during the norm-questioning 2016
election led to \emph{weaker} long-term support for radical-right
attitudes (nativism and authoritarianism) among Republicans and the
entire sample of respondents. This pattern is especially pronounced
among Republicans, suggesting a partisan asymmetry, albeit one in the
reverse direction of what was hypothesized. Rather than strengthening
support for radical-right attitudes, the experience of 2016 appears to
have either had no effect or to have even weakened support among some
groups. These findings have important implications on our understanding
of political socialization and the stickiness and adoption of elite
cues. At first, they might suggest that first-time voters are resistant
to the trickle-down adoption of radical-right attitudes; in fact, they
might even turn against them even more strongly.

However, given the broader context of existing literature, these
counter-intuitive findings might also admit at least two darker
interpretations. First, as Graham and Svolik
(\citeproc{ref-graham2020}{2020}) argue, ordinary citizens often express
broad opposition to (often undemocratic or illiberal) radical-right
attitudes when asked in the abstract, but this support may evaporate
when those principles conflict with partisan interests. Because the ANES
items used here were not embedded in a tradeoff or priming design, they
likely captured only nominal or superficial attitudes. A second
possibility, again drawing from Graham and Svolik
(\citeproc{ref-graham2020}{2020}), is that some Trump supporters may
genuinely reject his radical-right rhetoric, yet continue to support him
on policy grounds, thus generating a form of `retrospective backlash':
weaker radical-right attitudes despite (or, in the case of the
retrospective backlash, because of) exposure to such cues. In either
case, the implications are troubling. Whether due to shallow commitments
or conditional tolerance, these findings underscore a fragility in the
public's attachment to liberal democracy that that may not withstand
more explicit norm violation.

That said, several limitations temper the interpretation of these
results. Most notably, while some effects are statistically significant
and robust across subsamples, they weaken under alternative model
specifications, namely the addition of covariates controlling for
demographic characteristics. Additionally, the assignment variable used
to define voting eligibility (age in years ) is a blunt proxy that fails
to capture exact birthdates, introducing measurement error near the
cutoff. Furthermore, the limited internal consistency of the nativist
items (see appendix) further constrains the robustness of these
findings. Finally, this study truly examines medium- rather than long-
effects, as it studies attitudes only four years later than the first
exposure in 2016.

Future work could address these gaps in several ways. First, researchers
should replicate this design using datasets that include exact
birthdates, improving causal precision near the eligibility threshold.
Second, using more recent data (such as the 2024 ANES) could allow for
studying longer-term trends exploring different cohorts and political
cycles would help determine whether early exposure to far-right elite
cues produces lasting attitudinal change, or whether its effects fade
over time. Third, this study assumed that the voting eligibility itself
is enough to trigger this political `coming of age.' However, this also
includes a sizeable proportion of young adults who were eligible to vote
but chose not to. Future research could instead examine the effect of
first-time \emph{voting}, through a fuzzy RDD design that uses
eligibility as the instrumental variable. Finally, applying the same
logic to other political transitions, like Bolsonaro's election or the
AfD's stronger results in Germany since 2017, could reveal whether these
dynamics are context-dependent or broadly generalizable.

In sum, while these findings complicate our assumptions about the
downstream effects of the adoption of radical-right attitudes, they
might ultimately be reinforcing its most worrying implication: the
resilience of liberal democracy cannot be taken for granted, not even
among its newest members.

\newpage{}

\section{Works Cited}\label{works-cited}

\phantomsection\label{refs}
\begin{CSLReferences}{1}{0}
\bibitem[\citeproctext]{ref-anes}
ANES. 2021. {``American National Election Study 2020 Time Series Study
Full Release {[}Dataset and Documentation{]}.''}

\bibitem[\citeproctext]{ref-bekafigo2019}
Bekafigo, Marija A., Elena V. Stepanova, Brian A. Eiler, Kenji Noguchi,
and Kathleen L. Ramsey. 2019. {``The Effect of Group Polarization on
Opposition to Donald Trump.''} \emph{Political Psychology} 40 (5):
11631178. \url{https://www.jstor.org/stable/45204110}.

\bibitem[\citeproctext]{ref-calonico2025}
Calonico, Sebastian, Matias D. Cattaneo, and Max H. Farrell. 2025.
\emph{Rdrobust: Robust Data-Driven Statistical Inference in
Regression-Discontinuity Designs}.
\url{https://cran.r-project.org/web/packages/rdrobust/index.html}.

\bibitem[\citeproctext]{ref-clayton2021}
Clayton, Katherine, Nicholas T. Davis, Brendan Nyhan, Ethan Porter,
Timothy J. Ryan, and Thomas J. Wood. 2021. {``Elite Rhetoric Can
Undermine Democratic Norms.''} \emph{Proceedings of the National Academy
of Sciences} 118 (23): e2024125118.
\url{https://doi.org/10.1073/pnas.2024125118}.

\bibitem[\citeproctext]{ref-cremer2023}
Cremer, Tobias. 2023. {``A Europeanisation of American Politics?:
Trumpism and the Populist Radical Right in the United States.''}
\emph{Journal of Language and Politics} 22 (3): 396--414.
\url{https://doi.org/10.1075/jlp.22135.cre}.

\bibitem[\citeproctext]{ref-dinas2012}
Dinas, Elias. 2012. {``The Formation of Voting Habits.''} \emph{Journal
of Elections, Public Opinion and Parties} 22 (4): 431456.
\url{https://doi.org/10.1080/17457289.2012.718280}.

\bibitem[\citeproctext]{ref-dinas2013}
---------. 2013. {``Opening {``}Openness to Change{''}: Political Events
and the Increased Sensitivity of Young Adults.''} \emph{Political
Research Quarterly} 66 (4): 868882.
\url{https://doi.org/10.1177/1065912913475874}.

\bibitem[\citeproctext]{ref-druckman2024}
Druckman, James N. 2024. {``How to Study Democratic Backsliding.''}
\emph{Political Psychology} 45 (S1): 3--42.
\url{https://doi.org/10.1111/pops.12942}.

\bibitem[\citeproctext]{ref-franklin2004}
Franklin, Mark N. 2004. \emph{Voter Turnout and the Dynamics of
Electoral Competition in Established Democracies Since 1945}. Cambridge:
Cambridge University Press.
\url{https://doi.org/10.1017/CBO9780511616884}.

\bibitem[\citeproctext]{ref-ghitza2023}
Ghitza, Yair, Andrew Gelman, and Jonathan Auerbach. 2023. {``The Great
Society, Reagan's Revolution, and Generations of Presidential Voting.''}
\emph{American Journal of Political Science} 67 (3): 520--37.
\url{https://doi.org/10.1111/ajps.12713}.

\bibitem[\citeproctext]{ref-graham2020}
Graham, Matthew H., and Milan W. Svolik. 2020. {``Democracy in America?
Partisanship, Polarization, and the Robustness of Support for Democracy
in the United States.''} \emph{American Political Science Review} 114
(2): 392--409. \url{https://doi.org/10.1017/S0003055420000052}.

\bibitem[\citeproctext]{ref-jennings2014}
Jennings, M. Kent, and Richard G. Niemi. 2014. \emph{Generations and
Politics: A Panel Study of Young Adults and Their Parents}. Princeton
University Press.

\bibitem[\citeproctext]{ref-joel2014}
Joel, Samantha, Caitlin M. Burton, and Jason E. Plaks. 2014.
{``Conservatives Anticipate and Experience Stronger Emotional Reactions
to Negative Outcomes.''} \emph{Journal of Personality} 82 (1): 3243.
\url{https://doi.org/10.1111/jopy.12031}.

\bibitem[\citeproctext]{ref-knuckey2022}
Knuckey, Jonathan, and Komysha Hassan. 2022. {``Authoritarianism and
Support for Trump in the 2016 Presidential Election.''} \emph{The Social
Science Journal} 59 (1): 47--60.
\url{https://doi.org/10.1016/j.soscij.2019.06.008}.

\bibitem[\citeproctext]{ref-magleby2011}
Magleby, David B., Candice J. Nelson, and Mark C. Westlye. 2011. {``The
Myth of the Independent Voter Revisited.''} In, edited by Paul M.
Sniderman and Benjamin Highton, 0. Princeton University Press.
\url{https://doi.org/10.23943/princeton/9780691151106.003.0011}.

\bibitem[\citeproctext]{ref-mandel2014}
Mandel, David R., and Philip Omorogbe. 2014. {``Political Differences in
Past, Present, and Future Life Satisfaction: Republicans Are More
Sensitive Than Democrats to Political Climate.''} \emph{PLOS ONE} 9 (6):
e98854. \url{https://doi.org/10.1371/journal.pone.0098854}.

\bibitem[\citeproctext]{ref-mannheim1928}
Mannheim, Karl. 1928. \emph{Le problème des générations}. Paris: Armand
Colin.

\bibitem[\citeproctext]{ref-neundorf2013}
Neundorf, Anja, Kaat Smets, and Gema M García-Albacete. 2013.
{``Homemade Citizens: The Development of Political Interest During
Adolescence and Young Adulthood.''} \emph{Acta Politica} 48 (1): 92116.
\url{https://doi.org/10.1057/ap.2012.23}.

\bibitem[\citeproctext]{ref-taber2018}
Taber, Keith S. 2018. {``The Use of Cronbach{'}s Alpha When Developing
and Reporting Research Instruments in Science Education.''}
\emph{Research in Science Education} 48 (6): 1273--96.
\url{https://doi.org/10.1007/s11165-016-9602-2}.

\end{CSLReferences}

\newpage{}

\section{Appendix}\label{appendix}

For additional materials, such as details on the RDD design and index
construction, the specific items used for the construction of the
indices, and additional visualizations, please refer to the
\href{https://github.com/nikolaosvichos/Online-Appendix-Coming-of-Age-Under-Trump}{online
appendix.}

Link (in case hyperlink breaks):
\url{https://github.com/nikolaosvichos/Online-Appendix-Coming-of-Age-Under-Trump}




\end{document}
